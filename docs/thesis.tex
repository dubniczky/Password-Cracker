\documentclass[
	%parspace, % Térköz bekezdések közé / Add vertical space between paragraphs
	%noindent, % Bekezdésének első sora ne legyen behúzva / No indentation of first lines in each paragraph
	%nohyp, % Szavak sorvégi elválasztásának tiltása / No hyphenation of words
	%twoside, % Kétoldalas nyomtatás / Double sided format
	%draft, % Gyorsabb fordítás ábrák rajzolása nélkül / Quicker draft compilation without rendering images
	%final, % Teendők elrejtése / Set final to hide todos
]{elteikthesis}[2021/02/22]

% Packages
\usepackage[binary-units=true]{siunitx}
\usepackage{float}
\usepackage[final]{pdfpages}
\usepackage{multicol}
%\usepackage[pdftex]{graphicx}

\lstdefinelanguage{JavaScript}{
  keywords={typeof, new, true, false, catch, function, return, null, catch, switch, var, if, in, while, do, else, case, break},
  keywordstyle=\color{blue}\bfseries,
  ndkeywords={class, export, boolean, throw, implements, import, this},
  ndkeywordstyle=\color{darkgray}\bfseries,
  identifierstyle=\color{black},
  sensitive=false,
  comment=[l]{//},
  morecomment=[s]{/*}{*/},
  commentstyle=\color{purple}\ttfamily,
  stringstyle=\color{red}\ttfamily,
  morestring=[b]',
  morestring=[b]"
}

% Document's metadata
\title{Titkosított Jelszavak Feltörésének Gyorsítása GPU Parallelizációval} % Title
\date{2021} % Year of defense

% Author's metadata
\author{Nagy Richárd Antal}
\degree{programtervező informatikus BSc}

% Superivsor's metadata
\supervisor{Eichhardt Iván} % Internal supervisor's name
\affiliation{okatató, PhD} % Internal supervisor's affiliation

% University's metadata
\university{Eötvös Loránd Tudományegyetem} % University's name
\faculty{Informatikai Kar} % Faculty's name
\department{Algoritmusok és Alkalmazásaik\\ Tanszék} % Department's name
\city{Budapest} % City
\logo{elte_cimer_szines} % Logo

% Add bibliography file
\addbibresource{thesis.bib}

% The document
\begin{document}

% Set document language
\documentlang{magyar}

%\documentlang{english}

% List of todos (not in the final document)
%\listoftodos[\todolabel]

% Some document settings
\input{settings.tex}

% Title page (mandatory)
\maketitle
%\topicdeclaration


% Table of contents (mandatory)
\tableofcontents
\cleardoublepage

% Main content
\chapter{Bevezetés} % Introduction
\label{ch:intro}


% ------------------- Crypto basics
\section{A Kriptográfia Alapjai}

\subsection{Történet}
A fontos adatok titkosítása történelmünk során mindig fontos szerepet játszott akár háború közben a startégia terv biztonságos szállításához, vagy találmányok pontos specifikációjának a biztonságos tárolásához. Ennek köszönhetően a technológia fejlődésével a kriptográfiának mindig szorosan tartania kellett a lépést, hogy biztosítsa, hogy az adatokat nem csak most, hanem évek múlva sem lehet egyszerűen visszafejtenie illetékteleneknek \cite{katz2020introduction}.

A titkosítást és a visszafejtést legtöbbször papíron végezték, amely során egy előre meghatározott üzenetet a kriptográfus titkosított valamely eljárással, majd az eredményét leírta egy másik papírra. Ezek után a titkosított üzenet elküldésre került, amelyet optimális esetben kizárólag a céleszemély tudott visszafejteni. Ez azonban sok hibalehatőséget rejtett magában, hiszen amennyiben túl egyszerű a titkosítási módszer a feltörés is jelentősen könnyebb lett, azonban ha túl bonyolult akkor a kriptográfus hibáinak száma és egy üzenettel eltöltött ideje is megnőtt.

% Modern crypto
\subsection{Modern Kriptográfia}

Az első modern kriptográfus gépet két Dán üzletember fejlesztette ki 1915-ben, mely az Enigma nevet kapta \cite{ellis2005exploring}. A gép bemeneteként szolgált egy alap beállítás és az üzenetet, majd ebből visszaadott egy véletlenszerűenk tűnő karaktersorozatot. Megyegyező a beállításokkal és megegyező üzenettel a gép kimentene pontosan ugyan azt az eredményt adta minden alkalommal. Bárki a gép, az alap beállítás és a titkosított üzenet birtokában pontosan vissza tudta fejteni az eredi üzenetet. Azonban az alapbeállítás hiányában ez manuálisan közel lehetetlennek bizonyult.

A gép segíségével az összetett műveletek automatikusan megtörténtek és ezzel jelentősen csökkentették az emberhibából adódó problémákat, illetve sokszorosára gyorsították a folyamatot egy manuális számoláshoz képest.

Napjainkban ezen műveleteket már számítógépek végzik, amelyek az Enigma gép teljesítményének milliárdszorosát képesek elvégezni. Ennek köszönhetően a titkosítási módszerek teljesítményének is növekednie kellett.

% Methods
\subsection{Titkosítási módszerek}

Az általános titkosítási módszer úgynevezett szimmetrikus, vagy privát kulcsos titkosítás, amelyben egy kulcs és egy üzenet ismeretében egy titkosított üzenetet állítunk elő, amelyet a kulcs ismeretében lehet kizárólag visszafejteni. Fontos feltétel hogy a kulcs ismeretében gyorsan visszafejthető legyen, míg a kulcs ismerete nélkül szinte lehetetlen.

Ezzel szemben az asszimmetrikus módszer esetén két kulcsra van szükség. Egyre a titkosításhoz és egyre a feloldáshoz. Amennyiben az üzenetet az első kulccsal titkosítjuk, kizárólag a másodikkal lesz lehetőségünk visszafejteni azt. Ugyan ez működik az ellenkező irányban is. Ennek köszönhetően például az RSA algoritmus használatával egy publikus kulcsot bárhol tárolhatunk az interneten és amennyiben valaki üzenetet szeretne küldeni nekünk, titkosítja az üzenetét a publikus kulcsunkkal, majd elküldi nekünk és kizárólag mi tudjuk feloldani az üzenetet a privát kulccsal.

% Conclusion
\subsection{Konklúzió}

A kriptográfia egy komoly befolyással járó ágazat volt a történelmünk során, ugyanis esetenként akár országok jövője múlhatott azon hogy a titkos információkat biztonságosan tudják szállítani.
Az eddigiekben olyan titkosításról volt szó, amelyben valamilyen fél titkosít egy adatot és egy másik ezután vissza tudja nyerni pontosan az eredeti dokumentumot. Azonban erre nem minden esetben van szükség és bizonyos helyzetekben még hátrányos is. Ennek a megoldására alkalmazunk hash függvényeket.

% ------------------- Hash basics
\section{A Hash Alapjai}



% History
\subsection{Történet}

A szakdolgozatom fő inspirációjaként szolgált a számítógépes biztonságtechnika és a kódoptimalizálás témakörök vegyítése, amelyet a számítógéppel számolt hash feltörésére szolgáló programmal valósítottam meg.

A hash algoritmusokat egy egyszerűnek tűnő, de valójában összetett probléma megoldására alkották meg 1953-ban \cite{dang2015secure}. Tegyük fel, hogy van két nagy méretű dokumentum, amelyet meg akarunk vizsgálni, hogy megegyeznek-e. Mindezt egy olyan módszerrel amelynek nem kell végig vizsgálnia pontosan minden karaktert és elégséges a dokumentumok különbözőségének igazolására. Természetesen ezt teljes biztossággal kizárólag a pontos átvizsgálással lehet megtenni, azonban kialakíthatunk olyan módszereket ahol egyezés esetén annak az esélye hogy a dokumentumok nem egyeznek elenyésző. Sok egyéb tudományágban hagyatkozunk nem teljesen biztos, de valószínűségéből adódóan konklúzívnak vett eljárásokon. Például DNS teszt. Erre az informatikában egyik megoldásként született a hash használata.


% Good function
\subsection{Kritériumok}
\label{sec:criteries}

Egy hash algoritmus bemenetnek kap valamilyen adatot. Ez az algoritmustól függően tetszőleges bináris adatfolyam lehet, majd kimenetként visszaad egy fix hosszúságú kulcsot amelynek meg kell felelnie az alábbi alapszabályoknak \cite{preneel1993analysis}:

\begin{enumerate}
  \item \textbf{Univerzális}, azaz bármekkora és akármilyen típusú adatfolyamra működik, feltéve hogy reprezentálható bináris formában,
  \item \textbf{Adatvesztő}, azaz minden bemenetre azonos hoszúságú kimenetet ad, ezáltal a dokumentum nem lesz visszaállítható,
  \item \textbf{Determinisztikus}, azaz két azonos bemenet ugyan azt a kimenetet eredményezi,
  \item \textbf{Egyirányú}, azaz gyorsan kiszámolható a hash minden input alapján, viszont a kimenet ismeretében nem állítható vissza belátható időn belül a kiindulási dokumentum,
  \item \textbf{Egyedi}, azaz elenyésző az esélye, hogy két különböző bemenet azonos kimenetet generál,
  \item \textbf{Instabil}, azaz egy kisebb módosításnak a bemeneten nagy változást kell hogy eredményezzen a kimeneten.
\end{enumerate}

Egy ilyen algoritmus használatával két dokumentum összehasonlítható úgy, hogy mindkettőn lefuttatjuk ugyan azt az algoritmust és amennyiben a kimenet különbözik, a bemeneti dokumentumok bizotsan nem egyeznek. Azonban amennyiben egyeznek a hash-ek, szinte biztosan a dokumentumok is.

% SHAü256
\section{Az SHA-256 algoritmus}

Az SHA-256 egy kriptografikus hash algoritmus amely az SHA-2 egyik alváltozata, melyet az Egyesült Államokbeli National Security Agency fejlesztett ki 2001-ben~\cite{gueron2011sha}. Az SHA-2 két fő változattal rendelkezik: SHA-256 és SHA-512, amelyek között a fő különbség a használt szavak vagy szegmnesek mérete. Míg az SHA-256 32 bites, addig az SHA-512 64 bites szavakat használ. Egyik algoritmus sem feltörhető egyenlőre, ennek ellenére manapság a nagyobb rendszerek jobban preferálják ezen agloritmusok utódait az új rendszerekben. Integritás vizsgálathoz és kereséshez egyiket sem alkalmazzák viszonylag magas számítási igényük miatt, jelszavak titkosítására biztonságos tároláshoz az SHA-256 egy elterjedt módszer.

Az SHA-2 elődjére, az SHA-1-re épül, amely pedig a MerkleDamgård struktúrára. Abból adódóan, hogy ezt a struktúrát már sikerült feltörni, ezáltal az SHA-1 algoritmust is, bizonyítottá vált hogy csak idő kérdése, mielőtt az SHA-2 is hasonló sorsra jut \cite{sha2017survey}.

A megfelelő algoritmus kiválasztásánál három fő tulajdonságot vettem figyelembe:

\begin{enumerate}
  \item \textbf{Kritériumok:} Az algoritmusnak meg kell felelnie a \ref{sec:criteries} bekezdésben foglalt hash algoritmusokra vonatkozó feltételeknek,
  \item \textbf{Relevancia:} Az algoritmusnak jelszótitkosítás tekintetében napjainkban gyakran használtnak, vagy az algoritmus korábbi használatából adódóan jelentős mennyiségű feltörhető adattal kell rendelkezni,
  \item \textbf{Gyorsíthatóság:} Az algoritmusnak látványosan gyorsíthatónak kell lennie VGA parallelizáció segítségével
\end{enumerate}


% Unique proof
\subsection{Egyediség bizonyítása}

Az általam választott SHA-256 algoritmus \cite{gueron2011sha} minden bemenetre egy 64 hexadecimális karakteres (256 bites) kimenetet képez, amely tartalmazza az Angol ábécé betűit a-tól f-ig, illetve számjegyeket 0-tól 9-ig, így karakterenként 16 különböző lehetőségel rendelkezik. Kis és nagybetű között nem tesz különbséget, azonban a sorrend számít. Ez
%
{\hfil $$ (6 + 10)^{64} > \num{1.15e77} $$ \par}

különböző kulcslehetőséget képez. Ez olyan hatalmas mennyiség, hogy ha a következő 1000 évben a földön jelenleg élő minden ember minden nap elkészítene 100 egyedi kulcsú dokumentumot, akkor az összes lehetséges kulcs nagyjából 
%
\begin{equation}
    (1000 * \num{7800000000} * 365 * 100) = \num{2.847e17}
\end{equation}
\begin{equation}
    \frac{\num{2.847e17}}{(6 + 10)^{64}} \approx \num{2.46e-60}
\end{equation}
\begin{equation}
     = 0.000000000000000000000000000000000000000000000000000000000246 \%
\end{equation}
%
-a készülne csak el. Emiatt kijelenthetjük, hogy hash egyezés esetén feltételezhetjük hogy a bemeneti dokumentumok is pontosan megegyeznek.


% Use cases
\subsection{Alkalmazások}

A hash bevezetése sok problémára ad gyors és kényelmes megoldást, amelyek közül felsorolok egy néhány példát:


\begin{itemize}
  \item \textbf{Számítógéphálózat} átvitel során az átvitt adatok megsérülhetnek, illetve támadók módosíthatják azokat. Ezért az üzenetek mellé társítunk egy hash-et is, amelyet a küldő létrehoz, a fogadó pedig ellenőriz. \cite{chen2017approach}.
  \item \textbf{Adatbázis}-ból történő lekérdezés során az adatok hash algoritmusok felhasználásával csoportosíthatóak a gyorsabb kezelés érdekében \cite{dungan2010classifying}.
  \item \textbf{Adattárolás} esetében a merevlemezen a fájlok elérési útja, illetve a neve hash-ként van tárolva a gyorsabb beazonosítás érdekében \cite{garfinkel2010using}.
  \item \textbf{C++ modellek} esetén a nyelv alapkönyvtárában számos tárolási eszköz használ hash funkciókat az adatok kategorizálására és egyediségének vizsgálatára (pl: std::unordered\_set, std::unordered\_map) \cite{guntheroth2016optimized}.
  \item \textbf{Tranzakciókezelés} esetén egy limitként szolgálhat egy blokk kiszámítása, amelynek meg kell felelnie valamilyen sémának. Bitcoin esetén például 8 darab hexadecimális nulla karakterrel kell kezdődnie, akkor adható hozzá a tranzakció a blokk lánchoz (blockchain) \cite{courtois2014optimizing}.
  \item \textbf{Tartalom optimalizáció} esetén ha több felhasználó feltölti ugyan azt a médiát (kép, videó, hang), akkor a hash alapján be tudjuk azonosítani hogy az adott fájl már más által is felkerült-e, ebben az esetben az új feltöltő az új fájl helyett a régire fog mutatni és a másolat törlésre kerül.
  \item \textbf{Titkosítás} és autentikáció, azaz biztonságos azonosítás során Erről részletesebben a \ref{sec:hash_security} részben. 
\end{itemize}



% ------------------- Hash in security
\section{A Hash a Biztonságtechnikában}
\label{sec:hash_security}

\subsection{A Hash}

A számítógépes biztonságtechnikában a hash elsősorban jelszavak kapcsán jelenik meg, amelyeket szerverekre történő távoli bejelentkezéshez használunk. A megadott jelszót a szervernek valamely formában tárolnia kell, hogy sikeresen el tudja végezni a felhasználó azonosítását, azonban amennyiben ez az eredeti kulcs szöveges formájában történne meg, esetleges behatolás esetén minden regisztrált felhasználó jelszava megszerezhetővé válna. Ez veszélyes következményekkel járhat, ugyanis sokan egy jelszót több szolgáltatásnál is újra felhasználnak. Ezek tárolására egy általános megoldás a jelszó hash-algoritmussal történő egyirányú titkosítása.


\subsection{A Salt}

Ez a megoldás azonban egy problémát még nem old meg. Abból adódóan, hogy legtöbben egyszerű jelszavakat használnak, sokaknak ezek egyezni fognak és ezáltal az adatbázisban tárolt hozzárendelt hash-ek is. Emiatt annak ellenére hogy nem tudjuk mi a pontos jelszó, elég egy felhasználó jelszavát visszafejteni és a kapott kulcs biztosan működni fog minden másik egyező hash esetén is.

Erre megoldásként használjuk az úgynevezett "salt"-ot. Amely egy rövid karaktersorozat amit a jelszó titkosításakor véletlenszerűen generálunk, majd a jelszó végéhez fűzünk. Ez a salt végül a kiszámolt hash mellé kerül. A jelszó ellenőrzésekor a kapott jelszó után fűzzük ismét a hash mellett talált salt-ot és így futtatjuk le rajta az algoritmust. Ezáltal az azonos jelszavak is különbözően jelennek meg az adatbázisban.

A visszafejtést tovább komplikálhatjuk azzal, hogy többször is lefuttatjuk az algoritmust. Először a kulcs és salt kombináción, majd az ebből kapott hash eredményen. Ezt akármeddig ismételhetjük, azonban lineárisan növekszik a művelet ideje.

\subsection{Memóriaigény}

Az új generációs hash algoritmusok egyik fő szempontja a VGA gyorsítás megakadályozása. Ezt általában nagy mennyiségű memóriaigénnyel érik el, amely egy általános processzor számára, amely egyenletesen éri el a teljes memóriatartományt (dual-channel esetén a felét) nem okoz jelentős lassulást, azonban egy VGA esetén azt jelenti, hogy a lényegesen gyorsabb regiszter memória helyett a globális memóriát szükséges használni. Ennek köszönhetően esetenként a VGA-val történő feltörés lassabbnak is bizonyulhat mint ami egy hasonló árú processzorral elérhető \cite{mei2016dissecting}. Ezen oknál fogva ezek az algoritmusok kizárásra kerültek.

Példa a memóriaigény különbségekre:
\begin{table}[h]
\centering
    \begin{tabular}{|l|l|l|}
        \hline
        Algoritmus & Memóriaigény képlete & Memóriaigény \\
        \hline
        SHA-256 & 2 * 256 = \SI{512}{\bit} = & \SI{64}{\byte} \\
        \hline
        Scrypt & $N * 2r * \SI{64}{\byte} \rightarrow \num{16384} * 2*8 * \SI{64}{\byte}$ = & \SI{16}{\mega\byte} \\ 
        \hline
    \end{tabular}
    \caption{SHA-256 És Scrypt algoritmusok memóriaigényének összehasonlítása. Scrypt esetén a példaként felhozott N és r értékek egy általános konfigurációt mutatnak.}
\end{table}

\section{Motiváció}

A szakdolgozatom célja egy program elkészítése, amelyik egy jelszó hash-et egy lehetséges jelszó lista felhasználásával megpróbál visszafejteni videókártyás többszálú megoldással, illetve ezen program optimalizálása jelentős teljesítmény javulás érdekében. A programnak nem célja egy potenciális biztonsági rés kihasználására eszközt biztosítani, mindössze az általam kedvelt biztonságtechnika és optimalizálás témakörök vegyítésére ad lehetőséget.

A program elkészítéséhez a nyílt forráskódú OpenCL környezetet fogom használni, ugyanis kellő mélyséhű hozzáférést biztosít a grafikus kártya beslő kernelének kódolásához illetve a processzorral való kommunikációjához. Emellett elég általános ahhoz, hogy használható legyen több videókrátya rendszeren is.

A program írása során az általam elkészített rendszer potenciális gyorsítási lehetőségeket fogom tesztelni és elemezni, ezeket a saját számítógépem teljesítménye alapján rangsorolni. Egy véletlen jelszó feltörésének komplexitása alapján.
\cleardoublepage

\chapter{Felhasználói dokumentáció} % User guide
\label{ch:user}



% SW purpose
\section{A Szoftver Célja} 

A szoftver egy SHA-256 algoritmussal hash-elt jelszavak feltörésére alkalmas C++ és OpenCL-ben írt program. A célja a modern videókártya használat hatásának vizsgálata a jelszófeltörés sebességére nézve. Jelenleg a program képes videokártya használatával hash-elni egy jelszót, egy fájl megadásával hashelni több jelszót, vagy feltörni egy általános, vagy salt-al ellátott jelszót, amennyiben az szerepel a megadott ismert jelszavak listában. Ezt a feladatát speciálisan a használt rendszerre optimalizálva teszi.

A programot biztonságtechnikai kutatóknak, vagy diákoknak javaslom, akik érdekeltek abban, hogy milyen sebességgel törhető fel egy biztonságosnak számító jelszó a számukra elérhető hardverekkel.


\section{A Szoftver Használata}


%System requirements
\subsection{Rendszerkövetelmények}

A program Microsoft Windows operációs rendszeren futtatható exe formájában érhető el.

\begin{table}[H]
    \begin{tabular}{|l|l|l|}
        \hline
        & Minimális  & Optimális (vagy újabb/jobb) \\
        \hline
        Operációs Rendszer & Microsoft Windows 7 & Microsoft Windows 10 \\
        \hline
        Videókártya  & OpenCL-t támogató   & \begin{tabular}[c]{@{}l@{}}NVidia GeForce GTX 600-as széria\\
                                                            AMD Radeon R9/ HD 7000 széria
                                             \end{tabular}\\
        \hline
        Háttértár & 7200 RPM HDD & M.2 PCIe SSD \\
        \hline
    \end{tabular}
    \caption{A program rendszerkövetelményei.}
\end{table}


%Installation
\subsection{Telepítés}

A program letöltése után egy zip állomány áll a rendelkezésre. Ezt az állomány kell kicsomagolni egy erre alkalmas programmal (például 7-Zip) és a benne található mappát egy megfelelő helyre helyezni. Fontos, hogy a mappában található fájlok mindegyike továbbra is azonos mappában helyezkedjen el, ezen kívül bárhová helyezhető a számítógépen, ahol ön rendelkezik olvasási és futtatási joggal.

A program futtatása a konzolon keresztül a fájlokat tartalmazó mappából történik, ezért érdemes kényelmes elérést biztosítani a fájlhoz. Erre egy jó módszer lehet például az alapértelmezett C meghajtó gyökérkönyvtárába helyezni a mappát, amely a következő helyen elérhető lesz: C:\textbackslash gpucrack\textbackslash



%Contols
\subsection{Vezérlés}

A szoftver konzolos felületen indítási paraméterek megadásával konfigurálható. A szoftvert tartalmazó mappában megnyitott konzol ablakban a következő paranccsal tudjuk futtatni a programot (.\textbackslash sha256gpu.exe). Ekkor angol nyelven a program felsorolja a használható parancsokat, amelyek közül választhatunk:
%
\begin{itemize}
    \item platform - Felsorolja az elérhető platformokat és eszközöket.
    \item hash single <password> - Hash-el egy jelszót, majd kiírja az eredményt.
    \item hash single <password> <salt> - Hash-el egy jelszót egy salt-al együtt, majd kiírja az eredményt.
    \item hash multiple <source> <target> - Első paraméterként egy enterrel elválasztott kulcsokkal teli fájlt adhatunk meg, amely mindegyikét hashelni és enterrel elválasztva a target-nek meghatározott fájlba írja.
    \item crack single <source> <hash> - A megadott hash kódot megpróbálja a passwords fájlban található jelszavak segítségével feltörni. Amennyiben sikerül, kiírja az eredeti kulcsot, ellenkező esetben jelzi, hogy nem talált megfelelő kulcsot.
\end{itemize}

\noindent Ezek mellett minden parancsnak megadhatunk egy kapcsolót.
%
\begin{table}[H]
\centering
    \begin{tabular}{|l|l|l|}
        \hline
        Kapcsoló                          & Alapértelmezett & Hatás                                          \\
        \hline
        -p \textless{}id\textgreater{}    & 0                     & A használt platform kiválasztása               \\
        \hline
        -d \textless{}id\textgreater{}    & 0                     & A használt eszköz kiválasztása                 \\
        \hline
        -t \textless{}count\textgreater{} & 1024                  & Egyszerre feltörhető jelszavak száma \\
        \hline
        -k \textless{}size\textgreater{}  & 24                    & Egy bemeneti kulcs maximális mérete \\
        \hline
    \end{tabular}
    \caption{A programban használható kapcsolók leírása}
\end{table}

Minden kapcsoló használata opcionális, azonban a -t kifejezetten fontos az optimális sebesség elérésének érdekében, amelyet egy néhány teszt futtatással finomhangolni lehet. A -p és -d kapcsolók kizárólag akkor relevánsak, ha nem az alapértelmezett eszközt használjuk feltörésre. A kulcs maximális méretének csökkenése egy minimális teljesítmény növekedést eredményezhet, amennyiben ismert hogy, a maximum használt jelszó mérete például 16 volt és a mintadokumentum kizárólag ekkora vagy, ennél rövidebb jelszavakat tartalmaz.

\begin{figure}[H]
    \centering
    \includegraphics[width=\textwidth]{images/diagrams/use-case.png}
    \caption{Use-case diagram.}
    \label{fig:use-case}
\end{figure}





%Input
\subsection{Bemenet}

A bemeneti értékek minden parancsra különböznek, azonban ezek kategóriákra oszthatóak.
%
\begin{itemize}
    \item <password> jelszó: egy karaktersorozat, amely kizárólag az UTF-8 táblázat elemeit tartalmazhatja
    \item <salt> salt: a követelmény megegyezik a jelszóéval
    \item <source> fájl: egy fájlt egyértelműen meghatározó elérési útvonal. Lehet abszolút és relatív. A fájl tartalmának ASCII, vagy utf-8 karaktereknek kell lennie, amelyeket sortörés karakterek választanak el egymástól.
    \item <target> fájl: egy fájlt egyértelműen meghatározó elérési útvonal. Lehet abszolút és relatív. A fájl vagy nem létezik (a mappában kell rendelkezni írási joggal), vagy írható (a régi fájl felülírásra kerül). Ha a mappa nem írható az ön felhasználói profiljából, indítsa a programot rendszergazdaként!
    \item <hash> hash: 64 hexadecimális (0-9, a-f) karaktert tartalmazó szöveg. Amennyiben salt-al rendelkezik, azok a karakterek a hash előtt helyezkednek el hozzáfűzve ahhoz. Ilyen esetben lehet hosszabb a szöveg mint 64 karakter.
\end{itemize}
%
Amennyiben a program hibát észlel a bemenetekkel, azonnal megszakítja a futást és a felhasználó tudtára adja, hogy mi okozta azt.






%Output
\subsection{Kimenet}

A program a kimenetet többnyire az azt elindító konzolos ablakba írja. Ezzel együtt minden esetleges hibaüzenet, vagy sikertelen futás eredménye is oda kerül. A kimenetek ezen felül tartalmaznak részletes információt például a megkapott hash felbontásáról és a futásidőről.
%
\begin{itemize}
    \item platform: a számítógépen található platformok és eszközök listája
    \item hash sinle: a megadott jelszó, vagy jelszó és salt sha256 hash alakja
    \item hash multiple: a megadott jelszófájl jelszavainak a sha256 hash-jei a target fájlban.
    \item crack single: amennyiben sikeres volt a feltörés a hash-hez tartozó jelszó és egyéb információk, amennyiben nem volt sikeres, akkor ezt kiírja a felhasználónak.
\end{itemize}





%Examples
\subsection{Példák}

Paraméter nélküli indítás:
%
\begin{figure}[H]
    \centering
    \includegraphics[width=\textwidth]{images/examples/example-empty.png}
\end{figure}

Számítási hardware-platformok kiíratása:
%
\begin{figure}[H]
    \centering
    \includegraphics[width=\textwidth]{images/examples/example-platform.png}
\end{figure}

Hash kiszámítása:
%
\begin{figure}[H]
    \centering
    \includegraphics[width=\textwidth]{images/examples/example-hash.png}
\end{figure}

Jelszó sikeres feltörése:
%
\begin{figure}[H]
    \centering
    \includegraphics[width=\textwidth]{images/examples/example-crack.png}
\end{figure}

Jelszó sikertelen feltörése:
%
\begin{figure}[H]
    \centering
    \includegraphics[width=\textwidth]{images/examples/example-crack-unsuccessful.png}
\end{figure}

Jelszó paraméterezett feltörése:
%
\begin{figure}[H]
    \centering
    \includegraphics[width=\textwidth]{images/examples/example-crack-params.png}
\end{figure}

Jelszavak hashelése kimeneti fájlba:
%
\begin{figure}[H]
    \centering
    \includegraphics[width=\textwidth]{images/examples/example-hash-multiple.png}
\end{figure}

Jelszó és hash-fájlok tartalma (bal: bemenet, jobb: kimenet):
%
\begin{figure}[H]
    \centering
    \includegraphics[width=\textwidth]{images/examples/example-hash-files.png}
\end{figure}


\subsection{Hiba Esetén}

A program hiba esetén leáll és leírással próbálja segíteni a probléma diagnosztizálását. Amennyiben ismeretlen hiba történik, próbálja meg a következő lépéseket elvégezni:
%
\begin{enumerate}
    \item Ellenőrizze le, hogy a parancs megfelelő formátumú-e.
    \item Próbálja meg kapcsolók nélkül futtatni a programot.
    \item Indítsa el a programot a (platform) parancs segítségével és bizonyosodjon meg róla, hogy található legalább egy eszköz.
    \item Frissítse a használni kívánt eszközt a legújabb driver verzióra.
    \item Frissítse a C++ Runtime szolgáltatást a legújabb verzióra.
    \item Amennyiben fájlal dolgozik, bizonyosodjon meg róla, hogy az elérési út megfelelő, a fájlhoz van hozzáférése és az olvasható.
    \item Bizonyosodjom meg róla hogy az eredeti zip-ben tárolt fájlok mindegyike megtalálható a futtatható fájlal egy mappában.
    \item Futtassa a programot rendszergazdai jogosultsággal.
\end{enumerate}
\cleardoublepage

\chapter{Fejlesztői dokumentáció} % Developer guide
\label{ch:impl}

A tesztek futtatásához (amennyiben nincs egyéb meghatározva) egy asztali számítógépet használtam a következő paraméterekkel:

\begin{table}[h]
\centering
    \begin{tabular}{|l|l|l|}
        \hline
        \textbf{Megnevezés} & \textbf{Modell} & \textbf{Megjegyzés} \\
        \hline
        Alaplap & ASUS Prime X470-PRO & \\
        \hline
        CPU & Ryzen 7 2700X 8c/16t 4.00Ghz & alap órajel \\
        \hline
        RAM & Corsair Vengeance 2x8GB 2400Mhz DDR4 dual channel & \\
        \hline
        GPU & Nvidia Geforce GTX 1070 & alap órajel \\
        \hline
        SSD & Samsung 970 EVO 250GB & NVMe slot \\
        \hline
    \end{tabular}
    \caption{A tesztekhez használt számítógép paraméterei.}
\end{table}


% ------------------- OpenCL Basics
\section{Az OpenCL}



% Basics
\subsection{Alapok}


\begin{definition}
Heterogén rendszer: több processzor együttműködésén alapuló számítógépes rendszer.
\end{definition}


Az OpenCL egy nyílt forráskódú, cross-platform keretrendszer olyan homogén számítógépes környezetekhez, ahol a CPU és GPU vagy egyéb processzorok és ezek szálai párhuzamos együttműködésére van szükség. \cite{munshi2011opencl}.


\begin{definition}
Hoszt (host): Több processzoros rendszerek esetén azon eszköz, amely a többi irányításáért felel.
\end{definition}

\begin{definition}
Eszköz (device): Több processzoros rendszerek esetén azon eszköz, amely a hoszttól kapott feladatokat ellátja.
\end{definition}


A keretrendszer egy saját programozási nyelvvel rendelkezik, mely a C nyelvből fejlődött ki. Úgynevezett kerneleket tudunk írni, amelyekez az eszközökön tudunk futtatni. A projekt során a C++ API-t fogom alkalmazni, azonban sok nyelvhez elérhető hasonló könyvtár.

Az OpenCL keretrendszer inicializálásakor megadható hogy melyik eszközöket használjuk, illetve milyen szerepet fognak ezek betölteni. Szerepe alapján a hardvereket hoszt és eszköz kategóriákra bonthatjuk. Emellett fontos különbség az adott hardverek platform-ja. Minden platform különböző implementációját futtatja a az OpenCL keretrendszernek (pl.: Intel, AMD, Nvidia).

A program fordításának idejében nem ismerjük a jövőbeli felhasználó számítógépének pontos paramétereit, holott ez elengedhetetlen információ a fordító számára. Erre két megoldás közül tudunk választani:


\begin{enumerate}
  \item A kódot minden napjainkban használt eszközre optimalizálva lefordítjuk, illetve a későbbiekben amennyiben ez változik, új verziókat készítünk.
  \item Nem fordítjuk azonnal az eszközre szánt kernel kódot, ehelyett azt eltároljuk a program számára elérhető helyen, majd amikor a felhasználó először futtatja a programot, lefordítjuk a kernelt.
\end{enumerate}


Látszik hogy az első megoldás szinte kivitelezhetetlen, hiszen naprakészen tartani több ezer vagy akár tízezer konfigurációt lényegesen nagyobb költséggel jár, mint a felhasználó számára az első alkarommal várni akár kevesebb mint egy másodpercet hogy leforduljon a kód. A második megoldás hátránya azonban, hogy nem tudjuk elrejteni a kernel kódunkat és az könnyen másolható lesz. Ezzel szemben a kód futásidpőben szerkeszthető marad, amely tulajdonságát ki fogom használni a program során.


\lstset{caption={Tetszőleges méretű vektor összeadása OpenCL kernellel (forrás: Elte IK Computer Graphics GPGPU)}, label=src:cpp}
\begin{lstlisting}[language={C++}]
kernel void add(global const int* A,
                global const int* B,
                global int* C)
{ 
    int i = get_global_id(0); //Global ID in dimension 1 (0)
    C[i] = A[i] + B[i]; 
}
\end{lstlisting}



% Optimization
\subsection{Optimalizálás}

A szakdolgozat jelentős részét képzi az eszköz oldali parallel kód futásának optimalizálása. Ezen optimalizálás a lefordított kód elemzésével történhet legeffektívebben. Ehhez az AMD APP Kernel Anayzer-t használtam. Ez az eszköz a lefordított programkód utasításait visszafejti assebmly nyelvre. Az assembly kód betekintést adhat az esetleges optimálatlan kódrészekbe. 

Videókártyák esetén egy potenciális \SI{50}{\percent} lassítást jelenthet az elágazások használata. A futó szálak egyszerre egy parancsot képesek elvégezni, így aztán amíg bizonyos szálak egy elágazás hosszabb igaz oldalával dolgoznak, addig a többinek várni kell azokra még akkor is, ha gyorsabban végeztek a második felével. Ez alapján mindig elkerülöm a balanszolatlan hosszúságú elágazásokat, de igyekszem őket ahol csak lehetséges nélkülözni.

Szintén nagyobb teljesítményromálssal járhat a ciklusok használata. Egy ciklus minden lépése egy memóriacím növelésével, illetve egy elágazással jár. Emellett a cikluson belül felhasznált iterátor változót minden lépésnél fel kell használni akár műveletre (mely emiatt nem optimalizálható a fordíó által), vagy egy memóriacím lekérdezésére, amely az iterációs változó volatilitása miatt nem kerülhet a előre cache-be. Erre nyilvánvaló megoldást jelenthet az elkerülésük, vagy a ciklusok fixált lépésszámúvá tétele, majd a kiterítése. Pl:


\lstset{caption={A ciklus unroll előtt}, label=src:cpp}
\begin{lstlisting}[language={C++}]
#pragma unroll
for (int i = 0; i < 4; i++)
{
    keys[i] = (i + 1) * 10;
}
\end{lstlisting}


A \textbf{\#pragma unroll} kulcsszó jelzi a fordítónak, hogy a következő ciklus kiteríthető. Ez természetesen csak fordítás időben konstans hosszúságú ciklusok esetén alkalmazható.


\lstset{caption={A ciklus unroll után}, label=src:cpp}
\begin{lstlisting}[language={C++}]
keys[0] = 10;
keys[1] = 20;
keys[2] = 30;
keys[3] = 40;
\end{lstlisting}


A kiterített kód-ban előre látszik hogy a jelenlegi lépés során melyik mező lesz a következő, amelybe írni fogunk, ezért ez megoldható cachelés segítségével. Emellett az értékeket is műveletek helyett konstansokra tudja váltani a fordító.










% ------------------- SHA 256 Arithmetics
\section{Az SHA-256 Aritmetikája}

A hash kiszámolása közben több előre definiált műveletet alkalmazunk, amelyek együttes működése elősegíti a ténylegesen véletlenszerűnek tűnő eredmény előidézését.



% Defs
\subsection{Definíciók}


\begin{definition}
Logikai operátorok: AND, OR, XOR, NOT sorrendben a következő jelekkel feltüntetve: $\land, \lor, \oplus, \neg$.
\end{definition}

\begin{definition}
Integer összeadás: $A + B = A + B \bmod 2^{32} $. A modolás hatására a memóriaszektoron túlcsorduló bitek eltűnnek és minimum értékről kezdődik újra a számolás.
\end{definition}

\begin{definition}
Integer Bitshift: $A << N = A * 2^{N} \bmod 2^{32}$, vagy $A >> N = A / 2^{N} \bmod 2^{32}$. Az adott memóriaszektorban található bitek N darabszor balra vagy jobbra tolódnak. A szektoron kívülre eső biteket töröljük.
\end{definition}

\begin{definition}
Balanszolt bitművelet: Olyan bitműveletek, melyek interpretációjában azonos számú igaz és hamis szerepel. Példa balanszolt műveletre: XOR, NOT.
\end{definition}


\begin{table}[H]
    \centering
    \begin{tabular}{lll}
    
        \begin{tabular}{c||c}
            $\neg$ & \\
            \hline
            \hline
            1 & 0 \\
            \hline
            0 & 1 \\
        \end{tabular}
        &
        \begin{tabular}{c||c|c}
            $\land$ & 1 & 0 \\
            \hline
            \hline
            1 & 1 & 0 \\
            \hline
            0 & 0 & 0 \\
        \end{tabular}
        &
        \begin{tabular}{c||c|c}
            $\oplus$ & 1 & 0 \\
            \hline
            \hline
            1 & 0 & 1 \\
            \hline
            0 & 1 & 0 \\
        \end{tabular}
    \end{tabular}
    \caption{Három bináris függvény ereménye.}
\end{table}

A példákból látszik hogy nem minden bináris függvény esetén kapunk arányos mennyiségű igaz és hamis értéket. Ez természetesn azt okozza hogy n lépés esetén az eredmény konvergálni fog a magasabb esélyű értékhez.

\begin{table}[H]
\centering
    \begin{tabular}{|l|l||c|c||l|}
        \hline
        \multicolumn{2}{|c||}{\textbf{Művelet}} & \multicolumn{2}{c||}{\textbf{Kimenet}} & \multirow{2}{*}{\textbf{Balanszolt}} \\
        \textbf{Neve} & \textbf{Jele} & \textbf{1} & \textbf{0} &  \\
        \hline
        \hline
        NOT   &   $\neg$   &   \SI{50}{\percent}   &   \SI{50}{\percent}   &   igen \\
        \hline
        AND   &   $\land$   &   \SI{25}{\percent}   &   \SI{75}{\percent}   &   nem \\
        \hline
        OR   &   $\lor$   &   \SI{75}{\percent}   &   \SI{25}{\percent}   &   nem \\
        \hline
        XOR   &   $\oplus$   &   \SI{50}{\percent}   &   \SI{50}{\percent}   &   igen \\
        \hline
    \end{tabular}
    \caption{Bináris műveletek balanszoltságának összehasonlítása.}
\end{table}

Példaként válasszunk véletlenszerű 8 bites egész számokat: $[147, 71, 11, 156]$

\begin{table}[H]
\centering
    \begin{tabular}{|l|l|}
        \hline
        \textbf{Decimális} & \textbf{Bináris} \\
        \hline
        \hline
        $147$   &   $10010011$ \\
        \hline
        $71$   &   $01000111$ \\
        \hline
        $11$   &   $00001011$ \\
        \hline
        $156$   &   $10011100$ \\
        \hline
        \hline
        $\land$   &   $00000000$ \\
        \hline
        $\oplus$   &   $01000011$ \\
        \hline
    \end{tabular}
\end{table}


Látható, hogy már 4 művelet elvégzése után az $\land$ művelet balanszolatlansága következményeként a kimenet csupa hamisból áll. Ezzel szemben a $\oplus$ művelet esetén maradt 3 igaz bit, amely pontosan egy-el kevesebb mint a fele. A bemenet 32 bitje közül 15 igaz volt, ami szinén alulról közelíti a felét.




% Func and const
\subsection{Funkciók és Konstansok}

Az algoritmus a következő funkciókat fogja használni, melyek együttes működése balanszolt kimenetet ad:

\bigbreak

\begin{tabular}{rcl}

    $Shr(A, N)$     & = &   $(A >> N)$ \\
    $Rotr(A, N)$    & = &   $(A >> N) \lor (A << (32 - N)) $ \\
    $Ch(X, Y, Z)$   & = &   $(X \land Y) \oplus (\neg X \land Z)$ \\
    $Maj(X, Y, Z)$  & = &   $(X \land Y) \oplus (X \land Z) \oplus (Y \land Z)$ \\
    $\Sigma_0(X)$   & = &   $Rotr(X, 2)  \oplus Rotr(X, 14) \oplus Rotr(X, 22)$ \\
    $\Sigma_1(X)$   & = &   $Rotr(X, 6)  \oplus Rotr(X, 11) \oplus Rotr(X, 25)$ \\
    $\sigma_0(X)$   & = &   $Rotr(X, 7)  \oplus Rotr(X, 18) \oplus Rotr(X, 3)$  \\
    $\sigma_1(X)$   & = &   $Rotr(X, 17) \oplus Rotr(X, 19) \oplus Rotr(X, 10)$ \\

\end{tabular}


\bigbreak


Az algoritmus kiindulópontjaként véletlenszerű számokat kellett választani. A tényleges véletlenszerűség elengedhetetlen részét képezi szinte minden titkosító eljárásnak. Ezeket számítógép nem tudja előállítani valamilyen külső bemenet megadása nélkül. Például a random.org weboldala atmoszférikus zaj alapján generálja a véletlenszerű számokat. Ennél egy egyszerűbb módszer volt az első 64 prímszám köbgyökének a tört részét nek az első 32 bitjét venni.


\bigbreak


{\hfil $ \{ \;\; \lfloor \; (\sqrt[3]{i} \mod 1) * 2^{32} \; \rfloor \;\; | \;\; i \in P(64) \;\; \}  $, ahol $P(n)$ az első n prímszám \par}


\bigbreak


\begin{tabular}{lrll}

    $K_1$     &  $\sqrt[3]{2}   \approx$  & $1.25992104989$  &  $\xrightarrow{} \;\; 0x428a2f98$ \\
    $K_2$     &  $\sqrt[3]{3}   \approx$  & $1.44224957031$  &  $\xrightarrow{} \;\; 0x71374491$ \\
    $K_3$     &  $\sqrt[3]{5}   \approx$  & $1.70997594668$  &  $\xrightarrow{} \;\; 0xb5c0fbcf$ \\
    $K_4$     &  $\sqrt[3]{7}   \approx$  & $1.91293118277$  &  $\xrightarrow{} \;\; 0xe9b5dba5$ \\
    $K_5$     &  $\sqrt[3]{11}  \approx$  & $2.22398009057$  &  $\xrightarrow{} \;\; 0x3956c25b$ \\
    $K_n$     &  $\sqrt[3]{...} \approx$  & $...$            &  $\xrightarrow{} \;\; ...$ \\
    $K_{64}$  &  $\sqrt[3]{311} \approx$  & $6.77516895227$  &  $\xrightarrow{} \;\; 0xc67178f2$ \\

\end{tabular}


\bigbreak


Az eljárás végén a tömörítéshez szükség volt még további 8 darab 32 bites számra. Ezek értékét az első 8 prímszám négyzetgyökének a tört részének az első 32 bitjéből kapjuk. Ezek kiszámolására és ellenőrzésére használható a következő JavaScript kód:

\begin{algorithm}
    \lstset{caption={JavaScript kód a kiinduló hexadecimális számok kiszámolására.}, label=src:js}
    \begin{lstlisting}[language={JavaScript}]
    (() =>
    {
        [2,3,5,7,11,13,17,19].forEach((i) =>
           console.log(parseInt((Math.sqrt(i) % 1).toString(2).slice(2, 34), 2).toString(16)))
    })()
    \end{lstlisting}
\end{algorithm}

\begin{tabular}{lrll}

    $H_1$  &  $\sqrt{2}$  =  & $1.41421356237$  &  $\xrightarrow{} \;\;\; 0x6a09e667 $ \\
    $H_2$  &  $\sqrt{3}$  =  & $1.73205080757$  &  $\xrightarrow{} \;\;\; 0xbb67ae85 $ \\
    $H_3$  &  $\sqrt{5}$  =  & $2.23606797750$  &  $\xrightarrow{} \;\;\; 0x3c6ef372 $ \\
    $H_4$  &  $\sqrt{7}$  =  & $2.64575131106$  &  $\xrightarrow{} \;\;\; 0xa54ff53a $ \\
    $H_5$  &  $\sqrt{11}$ =  & $3.31662479036$  &  $\xrightarrow{} \;\;\; 0x510e527f $ \\
    $H_6$  &  $\sqrt{13}$ =  & $3.60555127546$  &  $\xrightarrow{} \;\;\; 0x9b05688c $ \\
    $H_7$  &  $\sqrt{17}$ =  & $4.12310562562$  &  $\xrightarrow{} \;\;\; 0x1f83d9ab $ \\
    $H_8$  &  $\sqrt{19}$ =  & $4.35889894354$  &  $\xrightarrow{} \;\;\; 0x5be0cd19 $ \\

\end{tabular}


\bigbreak




% Make blocks
\subsection{Blokkok Számítása}


A blokkáalakítás során a teljes üzenetet $n * 512$ bites méretűvé alakítjuk, ahol minden blokk 512 bites lesz. Egy optimalizálásként ezt a lépést részben kihagyhatjuk, hiszen feltételezzük, hogy a megkapott jelszó nem fogja felvenni az egy blokkban rendelkezésre álló 448 bitet (a maradék 64 az eredeti üzenet hossza), amely 56 byte, tehát 64 ascii, vagy 56 UTF-8 karakter tárolására képes. Így aztán egy blokkot készítünk a következőképpen:


\begin{enumerate}
    \itemsep-0.5em
    \item Az első bitek az megkapott kulcs karaktereinek a bitjei,
    \item a következő bit az üzenetet záró 1 bit,
    \item ezt követi k darab 0 bit, ahol $k = 448 - 1 - h$ (h : szöveg bitjeinek száma),
    \item az utolsó 64 bitre h azaz az üzenet hossza kerül 64 bites integerként.
\end{enumerate}


Az elkészített 512 bites blokkot egyből fel is bontjuk 16 darab 32 bites számra, amelyeket elhelyezünk egy W tömb első 16 elemeként. A maradékot a következő formulával számoljuk:

{\hfil $ W_i = \sigma_1(W_{i-2}) + W_{i-7} + \sigma_0(W_{i-15}) + W_{i-16} \;\;\; (16 < i \leq 64)$ \par}

Fontos hozzátenni, hogy az összeadások alatt az integer összeadást értjük.




% Hash tömörítés
\subsection{Hash Tömörítés}

Jelenleg a kód felbontásra került a W tömb elején, majd a további mezőit feltöltöttük módosított elemekkel a tömb elején kiindulópontként véve a $\sigma_0$ és $\sigma_1$ segítségével. Ezeket az értékeket vissza kell tömöríteni 256 bitre. A tömörítés közben a $\Sigma_0$, $\Sigma_1$, $Maj$ és $Ch$ műveleteket fogjuk használni, illetve az értékeket forgatni.

Beállítás:

{\hfil $ (a,b,c,d,e,f,g,h) = (H_1, H_2, H_3, H_4, H_5, H_6, H_7, H_8) $ \par}

64 kör iterálás, mely a következőből áll: $(1 \leq i \leq 64)$

\begin{table}[H]
    \centering
    \begin{tabular}{rcl}

        $T_1$  & = & $ h + \Sigma_1(e) + Ch(e,f,g) + K_i + W_i $ \\
        $T_2$  & = & $ \Sigma_0(a) + Maj(a,b,c) $ \\
          &  &  \\
        $h$  & = & $g$ \\
        $g$  & = & $f$ \\
        $f$  & = & $e$ \\
        $e$  & = & $d + T_1$ \\
        $d$  & = & $c$ \\
        $c$  & = & $b$ \\
        $b$  & = & $a$ \\
        $a$  & = & $T_1 + T_2$ \\
    
    \end{tabular}
\end{table}

Összefűzés:

{\hfil $ H = (H_1 + a) \cdot (H_2 + b) \cdot (H_3 + c) \cdot (H_4 + d) \cdot (H_5 + e) \cdot (H_6 + f) \cdot (H_7 + g) \cdot (H_8 + h)  $ \par}

\bigbreak

A $(\cdot)$ művelet jelen esetben bináris sorozatok összefűzését jelenti. Az összefűzés során a memóriaterület minden bitjét fejlhasználjuk, nem hagyjuk figyelmen kívül az első igaztól balra található hamisokat. A $H$ változó ebben az esetben egy 256 bites memóriaterületet takar.



% Output
\subsection{Kimenet}


A $H$ változó értéke konvertálható hexadecimális formába, majd visszamásolható a host memóriába hashelés esetén, vagy összehasonlítható egy előre megkapott 256 bites kulcssal ellenőrzés vagy feltörés esetén. Bitenkénti egyezés esetén elégségesen bizonyítható hogy hashelt kulcs megegyezik a másik hash elkészítésekor használttal, illetve az is hogy bitek kötötti különbség esetén a bementeki kulcsok nem egyeztek meg.

A hexadecimális felírási formához először 64 blokkra kell osztani a memóriaterületet, majd a blokkban található 4 bit méretű területekhez hozzárendelni a megfelelő hexadecimális karaktert a következő táblázat alapján


\begin{table}[H]
    \centering
    \begin{tabular}{cccc}

        \textbf{Bin}   &   \textbf{Dec}   &   \textbf{Hex}   &   \textbf{ASCII} \\
        0000   &   0    &   0   &   48 \\
        0001   &   1    &   1   &   49 \\
        0010   &   2    &   2   &   50 \\
        0011   &   3    &   3   &   51 \\
        0100   &   4    &   4   &   52 \\
        0101   &   5    &   5   &   53 \\
        0110   &   6    &   6   &   53 \\
        0111   &   7    &   7   &   53 \\
        1000   &   8    &   8   &   54 \\
        1001   &   9    &   9   &   55 \\
        1010   &   10   &   A   &   65 \\
        1011   &   11   &   B   &   66 \\
        1100   &   12   &   C   &   67 \\
        1101   &   13   &   D   &   68 \\
        1110   &   14   &   E   &   69 \\
        1111   &   15   &   F   &   70 \\
    
    \end{tabular}
    %\caption{safdsafdsafas}
    %\label{tab:djfhbgadjhn}
\end{table}


Ezt egyszerűen meg tudjuk tenni úgy, hogy ha 8 bites számokként tekintjük a 256 bites memóriaterületet, hiszen ez a legkisebb egyedileg címezgető terület, majd minden 8 bites számot logikai $\land$ művelettel éselünk az első vagy második felén csupa 1-ből álló 8 bites számmal. Pl:

Tegyük fel hogy az elsó 8 bites memóriaterület: $10110011$:


\begin{table}[H]
    \centering
    \begin{tabular}{rcccl}
    
        $ 10110011 $ & $\land$ & $ 00001111 $ & = & $0011 \xrightarrow{} 4 $ \\
        $ 10110011 $ & $\land$ & $ 11110000 $ & = & $1011 \xrightarrow{} B $ \\
        
    \end{tabular}
\end{table}


Tehát a hexadecimálisan felírható alak: \textbf{4B}. Ezt a folyamatot ismételhetjük 32-szer. Minden iteráció 2 karaktert eredményez, amely 64 karaktert jelent. A 64 hexadecimális karakter pedig megfelel az eredeti specifikációnak. Ezt a feladatot ellájtja a következő kódrész ezt a feladatot oldja meg azzal a különbséggel, hogy 8 bit méretű blokkok helyett a már rendelkezésre álló 32 bites blokkokon fog iterálni. Ez azonban az unroll miatt teljesen lineáris időben fut majd.


\lstset{caption={32 bites egész szám tömb hexadecimális karaktersorozattá konvertálása.}, label=src:cpp}
\begin{lstlisting}[language={C++}]
char hex_charset[] = "0123456789abcdef";
#pragma unroll
for (int i = 0; i < 8; i++)
{
    #pragma unroll
    for (int j = 8-1; j >= 0; hashInts[i] >>= 4, --j)
    {
        result[(i * 8) + j] = hex_charset[ hashInts[i] & 0xf ];
    }
}
result[64] = 0;
\end{lstlisting}


Az algoritmus bemeneteként megkapja a \textbf{hashInts} 8 darab 32 bites egész számot tartalmazó tömböt, és az ASCII Hexadecimális karaktereivé konvertált kimenetet elhelyezi a 65 byte méterű \textbf{result} karaktertömbben. A kimeneti tömb mérete a szöveg hosszánál egy byte-al nagyobb, hogy egy null karakter elférjen a string lezárására. A karakterek kiválasztását legegyszerűbben egy lookup táblázattal érjük el, ugyanis nem szekvenciálisak a kimenet karakterei. Több jelszó hashelése és fájlba írása esetén a null érték lezárás helyett egy enter (0x0D) karaktert használtam, így a kimenet azonnal írható lesz fájlba.


% Salt
\subsection{Salt}

A salt hozzáadásával a jelszótörés tovább komplikálódik, hiszen azonos jelszavak különböző salt használatával más ereményt generálnak. A salt egy publikus kulcsnak tekinthető, hiszen a végleges hash-hoz hozzáfűzve tároljuk és minden hashelésnél véletlenszerűen újat generálunk.

Példa:

\begin{table}[H]
    \centering
    \begin{tabular}{l|l|l}
        \textbf{Kulcs} & \textbf{Salt} & \textbf{Hash} \\
        \hline
        \hline
        banana &  & \begin{tabular}{@{}c@{}}b493d48364afe44d11c0165cf470a416 \\ 4d1e2609911ef998be868d46ade3de4e\end{tabular} \\
        \hline
        banana & Q9wvI9A & \begin{tabular}{@{}c@{}}50622ccfa4c8f58bd952b62f7fafe475 \\ 11fec498985921d6b13ac178cb413aee\end{tabular} \\
        \hline
        banana & Joz1BL1T & \begin{tabular}{@{}c@{}}7da2b105a959cff3b2c03c0c15fa11fa \\ 124636a21451eeead00cb7654c664f7e\end{tabular} \\
        
    \end{tabular}
    \caption{Azonos jelszó más hashet generál különböző kulcsok használatával.}
\end{table}

Ezeket a salt-okat minden crack esetén a kulcs után fűzzük és így végezzük el a hashelést. Ezek után a kimenet elé helyezzük. A kulcs hossza minden esetben $L - 64$, ahol $L$ az összefűzött hash és salt hossza.




% ------------------- Parallelization
\section{Párhuzamosítás}


%Parallelism
\subsection{Alapok}

A párhuzamosítás (parallelizáció) napjainkban az egyik fő módszere a számítógépek teljesítményének növelésére. Egyre nagyobb mennyiségű párhuzamosítható szálat tartalmaznak a mai processzorok, azomban ezek továbbra sem versenyképesek a videókártyák parallel terljesítményével, melyek akár százszor vagy ezerszer annyi műveletet képesek elvégezni, bár valamivel lassabban.

A videókártyák SIMD (Single Instruction, Multiple Data) programozási paradigmát használnak. Ez annyit jelent hogy a kártya minden szála egyszerre egy műveletet képes elvégezni, de azt sok szálon, melyek mindegykike külön adatokkal képes dolgozni.



% Data
\subsection{Adatmozgatás}

Az egyik fő limitáció az adatok mozgatása, melyet használat előtt be kell töltenünk az eszköz memóriájába, majd az eredményt visszamásolni a host memóriába. Ennek a sebességét sok paraméter befolyásolhatja. Emellett esetünkben a jelszavak feltöréséhez használt jelszó táblázatot is be kell másolnunk először a host memóriába lemezről, amely mégnagyobb limitációt jelent. Emiatt a projekt egyik fő kihívását ezen adatmozgatok parallel elvégzése fogja jelenteni.


\begin{figure}[H]
    \centering 
    \includegraphics[width=\textwidth]{images/pdf/data-movement.pdf}
    \caption{Adatmozgazás egy fájl tartalmának hashelése során.}
    \label{fig:lineargpu}
\end{figure}


Az ábrából látható, hogy amennyiben a hashelés közben történő adatmozgatást nem vesszük figyelembe, összesen 5 különböző alkalommal kell majd másolnunk. Ezen adatmásolások közül a legtöbbet az adatok beolvasása illetve kiírása fogja igényelni. A második leglassabb a host és az eszköz memória közötti mozgatás, így aztán ezeket próbálom a lehető legjobban parallelizálni. Ahogy befejeződött a merevlemezről történő beolvasás és az továbbításra került az eszköz felé, egyből kezdhetjük a következő adatbeolvasást a merevlemezről.


\begin{figure}[H]
    \centering 
    \includegraphics[width=\textwidth]{images/pdf/data-movement-parallel.pdf}
    \caption{Adatmozgazás parallel módon egy fájl tartalmának hashelése során.}
    \label{fig:parallelgpu}
\end{figure}



% Data
\subsection{Hashelés}

Az SHA256 algoritmus lépései egy hashen belül nem párhuzamosíthatóak. Az algoritmus így lett elkészítve, hiszen ha parallelizálható lenne akkor könnyebben feltörhető lennei, hiszen:
%
\begin{itemize}
    \item Egy hashelés részei gyorsíthatóak lennének, hiszen egy egy hashen sok szál tudna dolgozni egyszerre, ezzel exponenciálisan gyorsítva a feltörést,
    \item egy hashelés részekre bontható lenne, azaz bizonyos jelszavak részeinek kiszámolásához nem lenne szükség mindig előről újraszámolni részeket, hiszen azt egyszer már kiszámoltuk.
\end{itemize}
%
Több jelszó hashelése azonban nem függ egymástól, így az egyszerre futtatható feltörések számának kizárólag az eszköz és a beolvasás sebessége szab határt. A jelszavak parallel feltöréséhez szükség van egy fix méretű bemeneti és kimeneti bufferre, amelyen belül a szálak ki tudják számolni a pontos bemeneti és kimeneti pontjukat a következőképp:

Legyen:
\begin{itemize}
    \itemsep-0.5em
    \item $N$ : kulcsok száma
    \item $M$ : egy kulcs maximális mérete
    \item $P_0$ : bemeneti buffer kezdőpontja, mérete: $[N * M]$
    \item $P_1$ : kimeneti buffer kezdőpontja, mérete: $[N * 65]$
    \item $I : [0 .. N-1]$ jelenlegi szál indexe
\end{itemize}


ekkor a jelenlegi szál:


\begin{itemize}
    \itemsep-0.5em
    \item Bemenetének kezdete $ = P_0 + (I * M) $
    \item Bemenetének vége $ = P_0 + (I * M) + (M - 1) $
    \item Kimenetének kezdete $ = P_1 + (I * 65) $
    \item Kimenetének vége $ = P_1 + (I * 65) + 64 $
\end{itemize}

A kimeneti hash mérete 64 karakter, melyhez hozzájön egy null vagy enter.


\begin{figure}[H]
    \centering
    \includegraphics[width=\textwidth]{images/pdf/parallel-hashing.pdf}
    \caption{Szálak párhuzamosan dolgoznak a bemeneti kulcs bufferen és a kimeneti hash bufferen.}
\end{figure}




% Crack
\subsection{Feltörés}

Feltörés esetén azonban hashelni minden kulcsot, majd a kimenetüket kimásolni és a host-on végigiterálni egyezést keresve egy lassú és felesleges művelet lenne. Ezért ebben az esetben az első beolvasás előtt már bemásoljuk az eszköz kerneljébe a keresendő hash kódot és salt-ot és kizárólag egyezés esetén várunk választ az output bufferbe.

Legyen:
\begin{itemize}
    \itemsep-0.5em
    \item $ N $ : kulcsok száma
    \item $ M $ : egy kulcs maximális mérete
    \item $ S $ : a salt mérete
    \item $ P_0 $ : bemeneti buffer kezdőpontja, mérete: $[N * (M + S)]$
    \item $ P_1 $ : kimeneti buffer kezdőpontja, mérete egy 4 byte-os egész számnak felel meg
    \item $ I : [0 .. N-1]$ jelenlegi szál indexe
\end{itemize}


ekkor a jelenlegi szál:


\begin{itemize}
    \itemsep-0.5em
    \item Bemenetének kezdete $ = P_0 + (I * M) + (I * S) $
    \item Bemenetének vége $ = P_0 + (I * M) + (I * S) + (M + S - 1) $
    \item Kimenete $ = P_1 $, vagy nincs kimenet
\end{itemize}

Alapértelmezetten a kimenet bufferének értéke 0-ról indul. Ha a kimenet a futás után is nulla marad, akkor nem találtunk egyező kulcsot. Ezzel szemben amennyiben az érték megváltozott, akkor tudjuk hogy az adott azonosítójú szál oldotta meg sikeresen a visszafejtést és az értéket ki tudjuk olvasni az előbb betöltött memóriaterületről.


\begin{figure}[H]
    \centering
    \includegraphics[width=\textwidth]{images/pdf/parallel-cracking.pdf}
    \caption{Szálak párhuzamosan dolgoznak a bemeneti kulcs bufferen és a megadott hash és salt értékeken.}
\end{figure}










% ------------------- Optimization
\section{Optimalizálás}


A fejlesztési idő jelentős részét az optimalizálás töltötte ki. Miután volt egy programom, amely képes volt jelszavakat beolvasni és feltörni processzoron és videókártyán is egyből tudtam tesztelni a sebességet. A teszteléshez minden alkalommal egy 4 millió (\num{3735367}) elemű listát használt a program, melynek az utolsó elemeként szerepelt a helyes kulcs (ex-wethouder).
A futási idő méréséhez a C++ nyelv standard környezetében található chrono könyvtárat használtam, amely képes microsec pontossággal jelezni két utasítás között eltelt időt. Az végleges időkben kizárólag a tényleges feltöréssel töltött idő szerepel, nem tartalmazza az elején az inicializálsát és a fájl megnyitását, illetve a végén az eredmény kiiratását. Ezek ugyanis egyszer történnek csak meg, és tetszőlegesen nagy adattömeg esetén elenyésző a hatásuk.
A program debug mód kikapcsolásával és x64 architektúrára van építve, illeve a /O2 fordító paranccsal, amely többek között a kódoptimalizálást a futtatható fájl mérete helyett a sebességre fókuszálja.


\begin{definition}
Futási Stabilitás: Egy program futás időtartamának relatív eltérése több teszten keresztül azonos paraméterekkel, azonos hardveren és azonos alap kihasználtsággal.
\end{definition}

\begin{definition}
Hash per Second (h/s): Egy mértékegység amely egy algoritmus egy másodperc alatt elkészíthető hash kódjainak számára, egy adott hash algoritmus használatával, azonos hardveren és azonos alap kihasználtsággal.
\end{definition}




% First sim
\subsection{Első Szimuláció}

%2224092 cpu crack salted
%316579 gpu crack salted



Az alap tesztet 20 alkalommal futtattam CPU és GPU használatával is. Ezáltal egyrészt tisztán láthatjuk az egymáshoz képest számolt sebességkülönbséget, másrészt amennyiben a függvény futási ideje instabil, korrigálhatunk arra.


\begin{table}[H]
    \centering
    \begin{tabular}{l|l|l|l}
        \textbf{Eszköz} & \textbf{Salt} & \textbf{Futásidő} & \textbf{Teljesítmény}  \\
        \hline
        \hline
        \multirow{2}{*}{CPU} & Nem & $\num{105 658 927} \; \mu s \approx \num{106.7}s$ & $\approx \num{35 353} \; h/s$ \\
                             \cline{2-4}
                             & Igen & $\num{116 019 598} \; \mu s \approx \num{116.0}s$ & $\approx \num{32 196} \; h/s$ \\
                             \hline
                             
        \multirow{2}{*}{GPU} & Nem & $\num{21 250 652} \; \mu s \approx \num{21.3}s$ & $\approx \num{175 776} \; h/s$ \\
                             \cline{2-4}
                             & Igen & $\num{21 490 658} \; \mu s \approx \num{21.5}s$ & $\approx \num{173 814} \; h/s$ \\
    \end{tabular}
    \caption{?}
\end{table}


%amcharts.com
\begin{figure}[H]
    \centering
    \includegraphics[width=\textwidth]{images/charts/performance-1.png}
    \caption{CPU és GPU simán és salt-al történő összehasonlítása az első szimuláció során.}
\end{figure}

Látható a szimulációs eremdményekből, hogy a GPU feltörés sebessége jelenleg a CPU megfelelőjének majdnem 550\%-a. Továbbá megfigyelhető az is, hogy a hash alkalmazása nagyobb teljesítmény veszteséget okoz CPU-nál esetén (10\%), mint GPU esetén 1\%. Ez minden bizonnyal annak tudható be, hogy az utóbbinál a videókártyán történik a salt behelyezése a szöveg végére, így ez egy időben akár ezerszer is lefuthat.

A GPU-n salt használatával történő feltörés a projekt célja, ezért erre fókuszáltam futási stabilitás tesztelésénél. A teszt során 20 alkalommal futtattam a programot azonos körülmények között.



\begin{table}[H]
    \centering
    \begin{tabular}{rl}
        \begin{tabular}{r|l}
            \textbf{Iteráció} & \textbf{Futásidő} \\
            \hline
            \hline
            1.  & $\num{21 490 792} \mu s$ \\
            2.  & $\num{21 490 518} \mu s$ \\
            3.  & $\num{21 490 488} \mu s$ \\
            4.  & $\num{21 490 178} \mu s$ \\
            5.  & $\num{21 490 213} \mu s$ \\
            6.  & $\num{21 490 968} \mu s$ \\
            7.  & $\num{21 490 446} \mu s$ \\
            8.  & $\num{21 490 978} \mu s$ \\
            9.  & $\num{21 490 792} \mu s$ \\
            10. & $\num{21 490 534} \mu s$ \\
        \end{tabular}
        
        & 
        
        \begin{tabular}{r|l}
            \textbf{Iteráció} & \textbf{Futásidő} \\
            \hline
            \hline
            11. & $\num{21 490 768} \mu s$ \\
            12. & $\num{21 490 999} \mu s$ \\
            13. & $\num{21 490 703} \mu s$ \\
            14. & $\num{21 490 744} \mu s$ \\
            15. & $\num{21 490 776} \mu s$ \\
            16. & $\num{21 490 688} \mu s$ \\
            17. & $\num{21 490 688} \mu s$ \\
            18. & $\num{21 490 702} \mu s$ \\
            19. & $\num{21 490 182} \mu s$ \\
            20. & $\num{21 490 518} \mu s$ \\
        \end{tabular}\\
    \end{tabular}
    \caption{?}
\end{table}


\begin{figure}[H]
    \centering
    \includegraphics[width=\textwidth]{images/charts/performance-1-distribution.png}
    \caption{GPU Salt Feltörés Futásidő Eloszlása milliomod másodpercben számolva.}
\end{figure}


Ebből kiszámolható hogy az adatok szórása (standard deviation) $\sigma = 240\mu s$. Ilyen kis számoknál a programnyelv belső órájának pontossága is közrejátszhat az inkonzisztenciában, ezért a szórás a hibakorlátunk alatt helyezkedik el, tehát kijelenthetjük hogy a jelenlegi algoritmus futásideje stabil.


%CURRENT 3735367 compare in 0.308774


% Double Buffer
\subsection{Double Buffer}



Jelenleg az adatok beolvasása az \ref{fig:lineargpu} ábrának megfelelően zajlik, tehát megtörténik egy adatszegmens beolvasása, amely továbbításra kerül a feltörésre használt eszköz számára majd az eredmény megérkezését követően elkezdődik a következő beolvasás. Ezen természetesen tudunk javítani a  \ref{fig:parallelgpu} ábrának megfelelően. Erre egy dupla bufferezéses megoldást alkalmaztam, amely esetén a beolvasott adatoknak két egyforma méretű buffer lett létrehozva. Ezek legyenek $B_1, B_2, B_c, B_o \; (current, other)$
%
\begin{enumerate}
    \itemsep-0.5em
    \item $ B_c := B_1, B_o := B_2 $
    \item $ read(B_c) $
    \item $ crack(B_c) $ \& $ read(B_o) $
    \item $ swap(B_c, B_o) $
    \item back to 3.
\end{enumerate}

A bufferek méretének kiválasztását nem lehet fordításnál eldönteni, hiszen attól függnek, hogy a feltörésre használt rendszer adatok olvasása vagy az eszköz sebessége a kisebb keresztmetszet. Ezért a bufferek méretének megválasztásást a felhasználóra bízzuk, azomban a program mindenképpen választ magának egy alapértelmezett értéket, amennyiben egyéb utasítást nem kap. 

\begin{table}[H]
    \centering
    \begin{tabular}{l|l|l|r|l}
        \textbf{Eszköz} & \textbf{Futásidő} & \textbf{Teljesítmény} & \textbf{Különbség} & \textbf{Megjegyzés} \\
        \hline
        \hline
        
        CPU & $\num{116 019 598} \; \mu s \approx \num{116.0}s $ & $\approx \num{32 196} \; h/s$ & & \\
        \hline
                            
        GPU & $\num{21 490 658} \; \mu s \approx \num{21.5}s $ & $\approx \num{173 814} \; h/s$ & $+550\%$ & \\
        \hline
        
        GPU & $\num{20 201 236} \; \mu s \approx \num{20.2}s $ & $\approx \num{173 814} \; h/s$ & $+9\%$ & Double Buffer \\
        \hline
    \end{tabular}
    \caption{Teljesítmény táblázat salt-al és a double buffer hozzáadásával.}
\end{table}



% C IO
\subsection{C Bemenet}

Jelenleg a C++ eszközeit használom a fájl beolvasására. Ez a módszer std::string struktúrába másolja a fájl sorait, ahonnan később ki kell csomagolni és beleírni a bufferbe C alapú (null karakterrel lezárt) char* -ként.


\lstset{caption={Fájl sorainak beolvasása C++ fstream eszközökkel.}, label=src:cpp}
\begin{lstlisting}[language={C++}]
std::ifstream infile(fileName);

// ...

std::string line;
for (int i = 0; i < chunkSize && std::getline(infile, line); i++)
{
    strcpy(&currentBuffer[MAX_KEY_SIZE * i], line.c_str());
}

// ...

infile.close();
\end{lstlisting}

A példán az adatfolyan egy szegmensének beolvasása található. Ez a kód (5-9. sor) ismétlődik egészen addig, amíg elfogy a fájl, vagy az előző beolvasás feltörése sikeresen zárul. Látható hogy a beolvasás során a nyelv az adatokat egy std::fstream objektumon keresztül olvassa be, amelyet egy std::string-ben helyez el. Ezt a string-et végül egy null karakterrel terminált C string-re konvertáljuk és belemásoljuk a buffer megfelelő szegmensébe. Érezhető, hogy ezek felesleges extra műveletek, amelyek a futási idő egy jelentős részét jelenthetik.

Az optimalizáláshoz leváltottam a C++ nyelv által használt iostream és fstream eszközöket a standard C könyvtárra (stdio.h).

\lstset{caption={Fájl sorainak beolvasása C++ fstream eszközökkel.}, label=src:cpp}
\begin{lstlisting}[language={C}]
FILE* infile = fopen(fileName, "r");

// ...

for (int i = 0;
     i < chunkSize && fgets(&currentBuffer[MAX_KEY_SIZE * i], MAX_KEY_SIZE, infile) != NULL;
     i++)
     { }

// ...

fclose(infile);
\end{lstlisting}

Ebben az esetben látszik, hogy a fájlből történő beolvasás azonnal a bufferbe helyezi a szöveget. Egy hátrány, hogy a szövegek nincsenek lezárva null karakterekkel, hanem a sor beolvasója behelyezi a sorok végén található sortörés karaktert. Ezt minden szövegnél ki kell javítani hogy ne tekintse a hash algoritmus az entert is a jelszó részének (hiszen azok nem tartalmazhatjak sortörés karaktert). Ezt azonban a crack-et végző eszköz végzi, így módosítottam hogy a hashelés előtt az eszköz ellenőrzi a szöveg pontos hosszát úgy, hogy null vagy enter karakterig iterál, majd a végére fűzi a salt-ot.

\lstset{caption={Fájl sorainak beolvasása C++ fstream eszközökkel.}, label=src:cpp}
\begin{lstlisting}[language={C++}]
//Get key
uint length;
globalID = get_global_id(0);
globalKey = keys + globalID * KEY_LENGTH; //Get pointer to key
for (length = 0; length < KEY_LENGTH && (globalKey[length] != '\0' && globalKey[length] != '\n'); length++)
{
    key[length] = globalKey[length];
}

//Append salt
#pragma unroll
for (uint i = 0; i < SALT_LENGTH; i++)
{
    key[length + i] = XSTR(SALT_STRING)[i];
}
length += SALT_LENGTH;
key[length] = 0;
\end{lstlisting}

Látható, hogy a kulcs lokális memóriába történő másolása során a jelenlegi karakter hozzáadása előtt megnézzük, hogy a karakter null ($\setminus 0$), vagy sortörés-e. ($\setminus n$). Amennyiben igen, a kulcs végére értünk és a length értéket nem növeljük tovább.
\cleardoublepage

\input{chapters/sum.tex}
\cleardoublepage

% Appendices - useful for detailed information in long tables, many and/or large figures, etc.
\appendix
\input{appendices/sim.tex}
\cleardoublepage

% Irodalomjegyzék (kötelező)
% Bibliography (mandatory)
\addcontentsline{toc}{chapter}{\biblabel}
\printbibliography[title=\biblabel]
\cleardoublepage

% Ábrajegyzék (opcionális) - 3-5 ábra fölött érdemes
% List of figures (optional) - useful over 3-5 figures
\addcontentsline{toc}{chapter}{\lstfigurelabel}
\listoffigures
\cleardoublepage

% Táblázatjegyzék (opcionális) - 3-5 táblázat fölött érdemes
% List of tables (optional) - useful over 3-5 tables
\addcontentsline{toc}{chapter}{\lsttablelabel}
\listoftables
\cleardoublepage

% Forráskódjegyzék (opcionális) - 3-5 kódpélda fölött érdemes
% List of codes (optional) - useful over 3-5 code samples
\addcontentsline{toc}{chapter}{\lstcodelabel}
\lstlistoflistings
\cleardoublepage

% Jelölésjegyzék (opcionális)
% List of symbols (optional)
%\printnomenclature

\end{document}
